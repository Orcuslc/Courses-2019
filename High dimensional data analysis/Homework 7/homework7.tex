\documentclass{article}
\usepackage[a4paper,margin=1cm]{geometry}
\usepackage{fancyhdr}
\usepackage{extramarks}
\usepackage{amsmath}
\usepackage{amsthm}
\usepackage{amsfonts}
\usepackage{tikz}
\usepackage[plain]{algorithm}
\usepackage{algpseudocode}

\begin{document}
\author{Chuan Lu}
\title{BIOS:7600 Homework 7}
\maketitle

\medskip

\begin{enumerate}

\item Problem 5.1, Gaussian tail bound.

\begin{proof}
First, for any $t > 0$,
\begin{equation}
\begin{aligned}
\mathbb{P}(Z \ge \lambda) &\le \exp(-t\lambda)\mathbb{E}(e^{tZ}) = \frac{e^{-t\lambda}}{\sqrt{2\pi(\sigma^2/n)}}\int_{-\infty}^{\infty}e^{tz}e^{-\frac{z^2}{2\sigma^2/n}} dz \\
&= \frac{e^{-t\lambda}}{\sqrt{2\pi(\sigma^2/n)}}\int_{-\infty}^{\infty}\exp(-\frac{(z-\frac{\sigma^2}{n}t)^2}{2\sigma^2/n})\exp(\frac{\sigma^2 t^2}{2n})dz\\ 
&=\exp(-t\lambda + \frac{\sigma^2t^2}{2n}).
\end{aligned}
\end{equation}
Let
\begin{equation}
t = \frac{n\lambda}{\sigma} > 0,
\end{equation}
then the inequality becomes
\begin{equation}
\mathbb{P}(Z \ge \lambda) \le \exp(-\frac{n\lambda^2}{2\sigma^2}).
\end{equation}
By symmetry, 
\begin{equation}
\mathbb{P}(Z \le -\lambda) \le \exp(-\frac{n\lambda^2}{2\sigma^2}).
\end{equation}
Hence we get the result.
\end{proof}

\item Problem 5.3, Prediction bound under RE condition.

\begin{enumerate}
\item 
\begin{proof}
First, as proved in class, we have
\begin{equation}
\frac{1}{n}\lVert X\delta \rVert_2^2\le 3\lambda \sqrt{|\mathcal{S}|}\lVert \delta_S\rVert_2.
\end{equation}

By the restricted eigenvalue condition, we have 
\begin{equation}
\frac{1}{n}\delta^\top X^\top X\delta \ge \tau\lVert \delta\rVert_2^2\ge \tau\lVert \delta_S\rVert_2^2,
\end{equation}
so
\begin{equation}
\frac{1}{n}\lVert X\delta \rVert_2^2\le 3\lambda\sqrt{|\mathcal{S}|}\frac{1}{\sqrt{n\tau}}\lVert X\delta\rVert_2.
\end{equation}
Hence,
\begin{equation}
\frac{1}{n}\lVert X\delta \rVert_2^2 \le \frac{9}{\tau}\lambda^2|\mathcal{S}|.
\end{equation}
\end{proof}

\item 
\begin{proof}
Actually, by using the same equality in the last problem, we can get this result quickly.
\end{proof}
\end{enumerate}

\item Problem 6.1, Conditional distribution for random knockoffs.

\begin{proof}
By the conditional distributions of multivariate normal distribution,
\begin{equation}
(\tilde{x_i}\mid x_i = x) \sim N(((\Sigma-S)\Sigma^{-1}x)_i, (\Sigma-(\Sigma-S)\Sigma^{-1}(\Sigma-S))_{ii}) \sim N((x-S\sigma^{-1}x)_i, (2S-S\Sigma^{-1}S)_{ii}).
\end{equation}
\end{proof}

\item Problem 6.2, Selective inference in the p=2 case.
\begin{proof}
The condition becomes
\begin{equation}
|\frac{1}{n}x_2^\top (1-P_1)y + \lambda x_2^\top x_1(x_1^\top x_1)^{-1}| \le \lambda.
\end{equation}
By simplification,
\begin{equation}
-\lambda n(x_1^\top x_1 - x_2^\top x_1)\le (x_1^\top x_1x_2^\top - x_2^\top x_1x_1^\top)y \le \lambda n(x_1^\top x_1 - x_2^\top x_1).
\end{equation}
Then
\begin{equation}
A = \left[
\begin{aligned}
&x_1^\top x_1x_2^\top-x_2^\top x_1x_1^\top &0 \\
&0 &x_2^\top x_1x_1^\top - x_1^\top x_1x_2^\top,
\end{aligned}
\right],
\quad
b = \lambda n\left[
\begin{aligned}
&x_1^\top x_1 -x_2^\top x_1\\
&x_1^\top x_1 - x_2^\top x_1
\end{aligned}
\right]
,
\end{equation}
where 
$X = [x_1, x_2]$.

\end{proof}

\item Problem 6.3,  HIV drug resistance study.
\begin{enumerate}
\item 


\end{enumerate}
\end{enumerate}

\end{document}