
\documentclass{article}%
\usepackage{amsmath}
\usepackage{graphicx}
\usepackage{amsfonts}%
\usepackage{amssymb}


\setlength{\topmargin}{-0.75in}
\setlength{\textheight}{9.25in}
\setlength{\oddsidemargin}{0.0in}
\setlength{\evensidemargin}{0.0in}
\setlength{\textwidth}{6.5in}
\def\labelenumi{\arabic{enumi}.}
\def\theenumi{\arabic{enumi}}
\def\labelenumii{(\alph{enumii})}
\def\theenumii{\alph{enumii}}
\def\p@enumii{\theenumi.}
\def\labelenumiii{\arabic{enumiii}.}
\def\theenumiii{\arabic{enumiii}}
\def\p@enumiii{(\theenumi)(\theenumii)}
\def\labelenumiv{\arabic{enumiv}.}
\def\theenumiv{\arabic{enumiv}}
\def\p@enumiv{\p@enumiii.\theenumiii}
\pagestyle{plain}
\setcounter{secnumdepth}{0}
\newtheorem{theorem}{Theorem}
\newtheorem{acknowledgement}[theorem]{Acknowledgement}
\newtheorem{algorithm}[theorem]{Algorithm}
\newtheorem{axiom}[theorem]{Axiom}
\newtheorem{case}[theorem]{Case}
\newtheorem{claim}[theorem]{Claim}
\newtheorem{conclusion}[theorem]{Conclusion}
\newtheorem{condition}[theorem]{Condition}
\newtheorem{conjecture}[theorem]{Conjecture}
\newtheorem{corollary}[theorem]{Corollary}
\newtheorem{criterion}[theorem]{Criterion}
\newtheorem{definition}[theorem]{Definition}
\newtheorem{example}[theorem]{Example}
\newtheorem{exercise}[theorem]{Exercise}
\newtheorem{lemma}[theorem]{Lemma}
\newtheorem{notation}[theorem]{Notation}
\newtheorem{problem}[theorem]{Problem}
\newtheorem{proposition}[theorem]{Proposition}
\newtheorem{remark}[theorem]{Remark}
\newtheorem{solution}[theorem]{Solution}
\newtheorem{summary}[theorem]{Summary}
\newenvironment{proof}[1][Proof]{\textbf{#1.} }{\ \rule{0.5em}{0.5em}}

\begin{document}

\title{Theoretical Numerical Analysis, Assignment 1}
\author{Chuan Lu}
\date{\today}
\maketitle

\begin{enumerate}

\item Problem 1.4.13

Let 
\begin{equation}
f(x) = \left\{
\begin{aligned}
&-1, \quad& x \le -1 \\
&\frac{e^\frac{1}{x-1}-e^\frac{1}{x+1}}{e^\frac{1}{x-1}+e^\frac{1}{x+1}}, \quad& -1 < x < 1 \\
&1, \quad& x \ge 1
\end{aligned}
\right.
\end{equation}
Then by Exercise 1.4.11, we know $f\in C^\infty(\mathbb{R}) $.

\item Problem 1.5.4

First, by Jensen's inequality we have the given inequality. Then
\begin{equation}
\log\left(
\frac{a^p}{p}+\frac{b^q}{q}
\right)
\ge \frac{1}{p}\log a^p + \frac{1}{q}\log b^q = \log a + \log b = \log ab.
\end{equation}
Since $\log(\cdot) $ is an increasing function, 
\begin{equation}
\frac{a^p}{p}+\frac{b^q}{q} \ge ab.
\end{equation}

\item Problem 1.5.9

\begin{equation}
\begin{aligned}
|(f\ast g)(x)| &= \left|\int_{\mathbb{R}^d}f(y)g(x-y)dy\right| \le \int_{\mathbb{R}^d}|f(y)||g(x-y)|dy \\
&= \int_{\mathbb{R}^d}|f(y)||g(x-y)|^\frac{1}{p}|g(x-y)|^\frac{1}{q}dy \\
&\le \lVert f(y)g(x-y)^\frac{1}{p}\rVert_{L^p}\lVert g(x-y)^\frac{1}{q}\rVert_{L^q}.
\end{aligned}
\end{equation}
This comes from that 
\begin{equation}
fg^{\frac{1}{p}} \in L^p, \ g^{\frac{1}{q}} \in L^q.
\end{equation}

Then
\begin{equation}
\begin{aligned}
\lVert f\ast g\rVert_{L^p} &\le \left\lVert \lVert f(y)g(x-y)^\frac{1}{p}\rVert_{L^p}\lVert g(x-y)^\frac{1}{q}\rVert_{L^q} \right\rVert_{L^p}  \\
&\le \lVert f\rVert_{L^p} \left\lVert\lVert g^\frac{1}{p}\rVert_{L^p} \lVert g^{\frac{1}{q}}\rVert_{L^q} \right\lVert_{L^p} \\
&= \lVert f\rVert_{L^p} \lVert g\rVert_{L^1}^\frac{1}{p} \lVert g\rVert_{L^1}^\frac{1}{q}  = \lVert f\rVert_{L^p} \lVert g\rVert_{L^1}.
\end{aligned}
\end{equation}

\item Problem 1.5.10

Let
\begin{equation}
u(x) = u_n, \ v(x) = v_n,  \ n \le x < n+1.
\end{equation}
Then $u \in L^p, \ v\in L^q $, and 
\begin{equation}
\begin{aligned}
\sum_{n=0}^{\infty}u_nv_n &= \int_{\mathbb{R}} u(x)v(x)dx \le \int_{\mathbb{R}} |u(x)| |v(x)| dx \\
&\le \lVert u\rVert_{L^p}\lVert v\rVert_{L^q} = \lVert u\rVert_{\ell^p}\lVert v\rVert_{\ell^q}.
\end{aligned}
\end{equation}

\item Problem 1.5.12

First, consider $\Omega = [1, \infty), p = 1, q = 2$, and
$f(x) = \frac{1}{x} \in L^q $, but $f(x)\notin L^p $. Thus $L^p $ does not belong to $L^q $.

On the other hand, consider $\Omega = [1, \infty), p = 1, q = 2 $, and 
\begin{equation}
f(x) = \left\{
\begin{aligned}
&n, \quad n\le x < n+\frac{1}{n^3} \\
&0, \quad n+\frac{1}{n^3} \le x < n+1
\end{aligned}
\right.
\end{equation}
for each $n\in\mathbb{Z}$. Then $f\in L^p $ but $f\notin L^q $. Thus $L^q $ does not belong to $L^p $.

\end{enumerate}
\end{document}
87