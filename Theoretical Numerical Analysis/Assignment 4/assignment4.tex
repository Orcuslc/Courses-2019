
\documentclass{article}%
\usepackage{amsmath}
\usepackage{graphicx}
\usepackage{amsfonts}%
\usepackage{amssymb}


\setlength{\topmargin}{-0.75in}
\setlength{\textheight}{9.25in}
\setlength{\oddsidemargin}{0.0in}
\setlength{\evensidemargin}{0.0in}
\setlength{\textwidth}{6.5in}
\def\labelenumi{\arabic{enumi}.}
\def\theenumi{\arabic{enumi}}
\def\labelenumii{(\alph{enumii})}
\def\theenumii{\alph{enumii}}
\def\p@enumii{\theenumi.}
\def\labelenumiii{\arabic{enumiii}.}
\def\theenumiii{\arabic{enumiii}}
\def\p@enumiii{(\theenumi)(\theenumii)}
\def\labelenumiv{\arabic{enumiv}.}
\def\theenumiv{\arabic{enumiv}}
\def\p@enumiv{\p@enumiii.\theenumiii}
\pagestyle{plain}
\setcounter{secnumdepth}{0}
\newtheorem{theorem}{Theorem}
\newtheorem{acknowledgement}[theorem]{Acknowledgement}
\newtheorem{algorithm}[theorem]{Algorithm}
\newtheorem{axiom}[theorem]{Axiom}
\newtheorem{case}[theorem]{Case}
\newtheorem{claim}[theorem]{Claim}
\newtheorem{conclusion}[theorem]{Conclusion}
\newtheorem{condition}[theorem]{Condition}
\newtheorem{conjecture}[theorem]{Conjecture}
\newtheorem{corollary}[theorem]{Corollary}
\newtheorem{criterion}[theorem]{Criterion}
\newtheorem{definition}[theorem]{Definition}
\newtheorem{example}[theorem]{Example}
\newtheorem{exercise}[theorem]{Exercise}
\newtheorem{lemma}[theorem]{Lemma}
\newtheorem{notation}[theorem]{Notation}
\newtheorem{problem}[theorem]{Problem}
\newtheorem{proposition}[theorem]{Proposition}
\newtheorem{remark}[theorem]{Remark}
\newtheorem{solution}[theorem]{Solution}
\newtheorem{summary}[theorem]{Summary}
\newenvironment{proof}[1][Proof]{\textbf{#1.} }{\ \rule{0.5em}{0.5em}}

\newcommand{\lv}{\lVert}
\newcommand{\rv}{\rVert}

\begin{document}

\title{Theoretical Numerical Analysis, Assignment 4}
\author{Chuan Lu}
\date{\today}
\maketitle

\begin{enumerate}

\item Problem 3.5.3

Consider an homomorphism $[a, b]$ to $[-1, 1]$, where $x\to \frac{2x-a-b}{b-a}$. Then the polynomials
$$
\tilde{L}_n(x) = L_n(\frac{2x-a-b}{b-a})
$$
are the Legendre polynomials on $[a, b]$, where $L_n $ are the Legendre polynomials on $[-1, 1]$. 

Similarly, 
$$
\tilde{T}_n(x) = T_n(\frac{2x-a-b}{b-a})
$$
are the Chebyshev polynomials on $[a, b]$, where $T_n $ are the Chebyshev polynomials on $[-1, 1]$. The weight function should be
$$
\tilde{w}(x) = \frac{1}{\sqrt{1-(\frac{2x-a-b}{b-a})^2}}
$$

\item Problem 3.5.5

Let 
$$
xp_n(x) = \sum_{i=0}^{n+1}\alpha_i p_i(x), 
$$
then take inner product $(\cdot, \cdot)_{0, w} $ with $p_j $ on both sides,
$$
(xp_n, p_j) = \alpha_j\lVert p_j\rVert^2.
$$
For $j\le n-2$, $xp_j $ is a linear combination of $p_k$ for $0\le k\le n-1$,
$$
(xp_n, p_j) = \int xp_np_jw dx = \int p_n\sum_{k=0}^{n-1}\beta_k p_kw dx = \sum_{k=0}^{n-1}\beta_k\int p_n p_kwdx = 0.
$$
Then $\alpha_j = 0 $ for $j\le n-2$. So
$$
xp_n = \alpha_{n-1}p_{n-1} + \alpha_{n}p_n +\alpha_{n+1}p_{n+1},
$$
which is 
$$
p_{n+1}(x) = \frac{x-\alpha_n}{\alpha_{n+1}}p_n - \frac{\alpha_{n-1}}{\alpha_{n+1}}p_{n-1}.
$$

\item Problem 3.5.8

Let
$$
T_n(x) = \cos(n\arccos x) = 0,
$$
we have
$$
x = \cos(\frac{1}{n}(k\pi+\frac{1}{2}\pi)), \ k\in \mathbb{Z}, \ \frac{1}{n}(k+\frac{1}{2})\pi \in [-1, 1].
$$
Then $k = [-\frac{n}{\pi}-\frac{1}{2}, \frac{n}{\pi}-\frac{1}{2}]\cap \mathbb{Z}$.

Let
$$
T_n(x) = \cos(n\arccos x) = \pm 1,
$$
we have
$$
x = \cos(\frac{k\pi}{n}), \ k\in\mathbb{Z}, \ \frac{k\pi}{n}\in [-1, 1].
$$
Then
$k = [-\frac{n}{\pi}, \frac{n}{\pi}]\cap \mathbb{Z}$.

\item Problem 3.6.2

Consider $v \in V_1^\perp $ and $v_1\in V_1 $,
$$
(v, v_1) = \int_{-1}^1 v v_1 dx = \int_{0}^1vv_1dx = 0
$$
for all $v_1\in V_1 $. First, consider $v\in W = \{v\in V\mid v(x) = 0 \text{a.e. in} (0, 1) \}$, then each $v\in W$ satisties the condition above. On the other hand, if $\exists v\in V_1^\perp \setminus W $, consider $u(x)\in V_1 $, s.t.
$$
u(x) = \left\{
\begin{aligned}
&0, x\in (-1, 0], \\
&sgn(v(x)), x\in (0, 1),
\end{aligned}
\right.
$$
then since $v$ is not 0 a.e. on $(0, 1)$,
$(u, v) > 0$. Hence $V_1^\perp = W $.

\item Problem 3.7.5

$$
\begin{aligned}
\phi_j(x) &= \frac{2}{2n+1}D_n(x-x_j) = \frac{1}{2n+1}+\frac{2}{2n+1}\sum_{j=1}^{n}\cos(j(x-x_j)) \\
&= \frac{1}{2n+1} + \frac{2}{2n+1}\sum_{j=1}^{n}\cos jx_j\cos jx + \frac{2}{2n+1}\sum_{j=1}^n\sin jx_j \sin jx \in \mathbb{T}_n,
\end{aligned}
$$
and
$$
\phi_j(x_j) = \frac{1}{2n+1}+\frac{2}{2n+1}\sum_{i=1}^{n}1 = 1,
$$
$$
\phi_j(x_k) = 
$$


\end{enumerate}
\end{document}