
\documentclass{article}%
\usepackage{amsmath}
\usepackage{graphicx}
\usepackage{amsfonts}%
\usepackage{amssymb}


\setlength{\topmargin}{-0.75in}
\setlength{\textheight}{9.25in}
\setlength{\oddsidemargin}{0.0in}
\setlength{\evensidemargin}{0.0in}
\setlength{\textwidth}{6.5in}
\def\labelenumi{\arabic{enumi}.}
\def\theenumi{\arabic{enumi}}
\def\labelenumii{(\alph{enumii})}
\def\theenumii{\alph{enumii}}
\def\p@enumii{\theenumi.}
\def\labelenumiii{\arabic{enumiii}.}
\def\theenumiii{\arabic{enumiii}}
\def\p@enumiii{(\theenumi)(\theenumii)}
\def\labelenumiv{\arabic{enumiv}.}
\def\theenumiv{\arabic{enumiv}}
\def\p@enumiv{\p@enumiii.\theenumiii}
\pagestyle{plain}
\setcounter{secnumdepth}{0}
\newtheorem{theorem}{Theorem}
\newtheorem{acknowledgement}[theorem]{Acknowledgement}
\newtheorem{algorithm}[theorem]{Algorithm}
\newtheorem{axiom}[theorem]{Axiom}
\newtheorem{case}[theorem]{Case}
\newtheorem{claim}[theorem]{Claim}
\newtheorem{conclusion}[theorem]{Conclusion}
\newtheorem{condition}[theorem]{Condition}
\newtheorem{conjecture}[theorem]{Conjecture}
\newtheorem{corollary}[theorem]{Corollary}
\newtheorem{criterion}[theorem]{Criterion}
\newtheorem{definition}[theorem]{Definition}
\newtheorem{example}[theorem]{Example}
\newtheorem{exercise}[theorem]{Exercise}
\newtheorem{lemma}[theorem]{Lemma}
\newtheorem{notation}[theorem]{Notation}
\newtheorem{problem}[theorem]{Problem}
\newtheorem{proposition}[theorem]{Proposition}
\newtheorem{remark}[theorem]{Remark}
\newtheorem{solution}[theorem]{Solution}
\newtheorem{summary}[theorem]{Summary}
\newenvironment{proof}[1][Proof]{\textbf{#1.} }{\ \rule{0.5em}{0.5em}}

\newcommand{\lv}{\lVert}
\newcommand{\rv}{\rVert}

\begin{document}

\title{Theoretical Numerical Analysis, Assignment 2}
\author{Chuan Lu}
\date{\today}
\maketitle

\begin{enumerate}

\item Problem 3.1.5

First, as $f$ is continuous, we have
$$
\lim_{t\to 0^+}f(-\log\frac{t}{a}) = \lim_{t\to\infty}f(t) = 0.
$$
Thus 
$$
g(t) = \left\{
\begin{aligned}
&f(-\log t/a), \quad & 0 < t \le 1, \\
&0, \quad & t = 0,
\end{aligned}
\right.
$$
is continuous. By Weierstrauss Approximation Theorem, for $\epsilon_n = \frac{1}{n} $, there is a polynomial $p$ of degree $d_n $, s.t.
$$
\lVert f(-\log t/a) - p(t)\rVert_\infty \le \epsilon_n.
$$
Let $x = -\log t/a$, then this is just 
$$
\lVert f(x) - \sum_{j=0}^{d_n}c_{n, j}e^{-jax}\rVert_\infty \le \epsilon_n.
$$
Then we can construct a list of desired funtions in a ``stepwise'' way, i.e., let $c_{m, j} = 0 $ if $d_{n-1} < m \le d_n$ for $q_n $.

\item Problem 3.2.4

We can see that 
$$
V_n(x_i) = 0, \quad 0\le i\le n-1.
$$
Also, we can notice that $V_n $ is a polynomial with degree up to $n$, so
$$
V_n = c_n(x-x_0)\cdots(x-x_{n-1}).
$$
The coefficient of the highest term is 
$$
c_n = \det\left(
\begin{matrix}
1 & x_0 & \cdots & x_0^{n-1} \\
\hdots & \hdots & \hdots & \hdots \\
1 & x_{n-1} & \cdots & x_{n-1}^{n-1}
\end{matrix}
\right)
= V_{n-1}(x_{n-1}).
$$
Thus, 
$$
\begin{aligned}
V_n(x_n) &= V_{n-1}(x_{n-1})(x_n-x_0)\cdots(x_n-x_{n-1}) \\
&= V_{n-2}(x_{n-2})(x_{n-1}-x_0)\cdots(x_{n-1}-x_{n-2})(x_n-x_0)\cdots(x_n-x_{n-1}) \\
&= \cdots \\
&= \prod_{j>i}(x_j-x_i).
\end{aligned}
$$

\item Problem 3.3.8

By the properties of inner products, for $\lambda \in [0, 1]$ and $u, v\in V$,
$$
\begin{aligned}
f(\lambda u+(1-\lambda)v) &= (\lambda u+(1-\lambda)v, \lambda u+(1-\lambda)v) \\
&= \lambda^2 \lVert u\rVert^2 + (1-\lambda)^2\lVert v\rVert^2 + 2\lambda(1-\lambda)(u, v) \\
&\le\lambda^2\lVert u\rVert^2+(1-\lambda)^2 \lVert v\rVert^2 + 2\lambda(1-\lambda)\lVert u\rVert\lVert v\rVert \\
&=\lambda \lVert u\rVert^2+(1-\lambda)\lVert v\rVert^2-\lambda(1-\lambda) (\lVert u\rVert^2 + \lVert v\rVert^2 - 2\lVert u\rVert \lVert v\rVert) \\
&= \lambda f(u)+(1-\lambda)f(v)-\lambda(1-\lambda)(\lVert u\rVert-\lVert v\rVert)^2\\
&\le \lambda f(u)+(1-\lambda)f(v).
\end{aligned}
$$
When $u\ne v$, the last inequality is strict. Hence $f$ is strictly convex.

\item Problem 3.3.9

Suppose 
$$
\lv u+v\rv = \lv u\rv+\lv v\rv,
$$
then
$$
\lv u\rv^2+\lv v\rv^2+2\lv u\rv\lv v\rv = \lv u\rv^2+\lv v\rv^2+2(u,v) \le \lv u\rv^2+\lv v\rv^2+2\lv u\rv\lv v\rv.
$$
By Cauchy-Schwarz inequality, 
$$
u = v.
$$
Then the inner product space is strictly normed.

\item Problem 3.4.10

$$
p_n(x) = \sum_{i=0}^{n}(f, \phi_i)\phi_i = \sum_{i=0}^n (f, \cos jx)\cos jx+\sum_{i=1}^{n}(f, \sin jx)\sin jx,
$$
and
$$
(f, \cos jx) = \int_{0}^{2\pi} f(x)\cos(jx) dx, \ (f, \sin jx) = \int_{0}^{2\pi} f(x)\sin(jx)dx.
$$
Particularly,
$$
(f, 1) = \frac{1}{2\pi}\int_0^{2\pi}f(x)dx.
$$
So the formula (3.4.9) is derived.

For the Parseval's equality, 
$$
\lv f\rv_{L^2(0, 2\pi)}^2 = \frac{1}{2}a_0^2+\sum_{i=1}^{\infty}(a_j^2+b_j^2).
$$

\end{enumerate}
\end{document}