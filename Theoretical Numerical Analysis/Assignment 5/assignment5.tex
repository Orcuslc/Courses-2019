
\documentclass{article}%
\usepackage{amsmath}
\usepackage{graphicx}
\usepackage{amsfonts}%
\usepackage{amssymb}


\setlength{\topmargin}{-0.75in}
\setlength{\textheight}{9.25in}
\setlength{\oddsidemargin}{0.0in}
\setlength{\evensidemargin}{0.0in}
\setlength{\textwidth}{6.5in}
\def\labelenumi{\arabic{enumi}.}
\def\theenumi{\arabic{enumi}}
\def\labelenumii{(\alph{enumii})}
\def\theenumii{\alph{enumii}}
\def\p@enumii{\theenumi.}
\def\labelenumiii{\arabic{enumiii}.}
\def\theenumiii{\arabic{enumiii}}
\def\p@enumiii{(\theenumi)(\theenumii)}
\def\labelenumiv{\arabic{enumiv}.}
\def\theenumiv{\arabic{enumiv}}
\def\p@enumiv{\p@enumiii.\theenumiii}
\pagestyle{plain}
\setcounter{secnumdepth}{0}
\newtheorem{theorem}{Theorem}
\newtheorem{acknowledgement}[theorem]{Acknowledgement}
\newtheorem{algorithm}[theorem]{Algorithm}
\newtheorem{axiom}[theorem]{Axiom}
\newtheorem{case}[theorem]{Case}
\newtheorem{claim}[theorem]{Claim}
\newtheorem{conclusion}[theorem]{Conclusion}
\newtheorem{condition}[theorem]{Condition}
\newtheorem{conjecture}[theorem]{Conjecture}
\newtheorem{corollary}[theorem]{Corollary}
\newtheorem{criterion}[theorem]{Criterion}
\newtheorem{definition}[theorem]{Definition}
\newtheorem{example}[theorem]{Example}
\newtheorem{exercise}[theorem]{Exercise}
\newtheorem{lemma}[theorem]{Lemma}
\newtheorem{notation}[theorem]{Notation}
\newtheorem{problem}[theorem]{Problem}
\newtheorem{proposition}[theorem]{Proposition}
\newtheorem{remark}[theorem]{Remark}
\newtheorem{solution}[theorem]{Solution}
\newtheorem{summary}[theorem]{Summary}
\newenvironment{proof}[1][Proof]{\textbf{#1.} }{\ \rule{0.5em}{0.5em}}

\newcommand{\lv}{\lVert}
\newcommand{\rv}{\rVert}

\begin{document}

\title{Theoretical Numerical Analysis, Assignment 5}
\author{Chuan Lu}
\date{\today}
\maketitle

\begin{enumerate}

\item Problem 4.1.4

(a)
$$
\begin{aligned}
S_nf(x) &= \frac{1}{2\pi}\int_{-\pi}^{\pi}f(x)dx + \sum_{j=1}^{n}\frac{1}{\pi}\int_{-\pi}^\pi f(x)\cos(jx)dx \cos(jx) + \frac{1}{\pi}\int_{-\pi}^\pi f(x)\sin(jx)dx \sin(jx) \\
&= \frac{1}{2\pi}\int_{-\pi}^{\pi}f(x)dx + \sum_{j=1}^{n}\frac{1}{\pi}\int_{-\pi}^{\pi} f(t)(\cos(jt)\cos(jx)+\sin(jt)\sin(jx))dt \\
&= \frac{1}{2\pi}\int_{-\pi}^{\pi}f(x)dx + \sum_{j=1}^{n}\frac{1}{\pi}\int_{-\pi}^{\pi} f(t)\cos(j(t-x))dt \\
&= \int_{-\pi}^{\pi}f(t)K_n(t-x)dt.
\end{aligned}
$$

(b)
$$
\left.\int_{-\pi}^{0} K_n(t)dt = \int_{-\pi}^0\frac{1}{2\pi} + \frac{1}{\pi}\sum_{i=1}^{n}\cos(jt)dt = \frac{1}{2} + \frac{1}{\pi}\sum_{i=1}^{n}\frac{1}{j}\sin(jt)\right|_{-pi}^0 = \frac{1}{2}.
$$
Since $K_n $ is  an even function, we know $\int_{0}^{\pi}K_n(t)dt = \frac{1}{2} $.

When $t = 2k\pi$, we know $K_n(t) = \frac{1}{2\pi}+\frac{n}{\pi} $; When $t \ne 2k\pi$, 
$$
k_n(t) = \frac{1}{2\pi \sin(t/2)}(\sin(t/2)+\sum_{j=1}^{n}(\sin((j+\frac{1}{2})t) - \sin((j-\frac{1}{2})t))) = \frac{1}{2\pi\sin(t/2)}\sin((n+1/2)t)
$$

(c), (d)
This are trivial.


\item Problem 4.2.1

(4.2.5) is trivial since $\mathcal{F}$ is linear.

(4.2.6)
$$
\lVert \mathcal{F}(f)\rVert_{L^\infty} \le \frac{1}{(2\pi)^{d/2}}\int_{\mathbb{R}^d} |f|dx = (2\pi)^{-d/2}\lVert f\rVert_{L^1}.
$$

(4.2.7)
This can be proven by $|\alpha|$ integration by parts, and using the property that $\partial f \to 0$ as $x\to\infty$.

(4.2.8)
This also can be proven by $|\alpha|$ integration by parts.

\item Problem 4.2.8

Let $g(x) = \overline{f(-x)}, h = f\ast g$. Then
$$
\mathcal{F}(h) = \mathcal{F}(f)\mathcal{F}(g) = \mathcal{F}(f)^2.
$$
Since
$$
h(0) = \int_{\mathbb{R}^d}f(x)g(-x)dx = \lVert f\rVert_{L^2}^2,
$$
$$
h(0) = (2\pi)^{-d/2}\int_{R^d}\mathcal{F}(h)(y)dy = \lVert \mathcal{F}(f)\rVert_{L^2}^2,
$$
we have $ \lVert \mathcal{F}(f)\rVert_{L^2}^2 = \lVert f\rVert_{L^2}^2 $.

\item Problem 4.3.2

Notice
$$
|\hat{y_k}|^2 = \hat{y_k}\overline{\hat{y_k}} = \sum_{j=0}^{n-1}y_j\sum_{l=0}^{n-1}y_l \exp(k(j-l)2\pi i/n),
$$
then
$$
\begin{aligned}
\sum_{k=0}^{n-1}|\hat{y_k}|^2 &= \sum_{k=0}^{n-1}\sum_{j=0}^{n-1}y_j\sum_{l=0}^{n-1}y_l \exp(k(j-l)2\pi i/n) \\
&= \sum_{j=0}^{n-1}y_j\sum_{l=0}^{n-1}y_l\sum_{k=0}^{n-1}\exp(k(j-l)2\pi i/n) \\
&= \sum_{j=0}^{n-1}y_j\sum_{l=0}^{n-1}y_l\frac{1-\exp((j-l)2\pi i )}{1-\exp((j-l)2\pi i/n )} \\
&= n\sum_{j=0}^{n-1}y_j^2
\end{aligned}
$$


\item Problem 4.3.3

For $0\le k < n$,
$$
(F_{2n}y)_k = (F_ny_e)_k + (D_nF_ny_o)_k = \mathcal{F}_n(\{y_{2j}\})_k + w_{2n}^{-k}\mathcal{F}_n(\{y_{2j+1}\})_k,
$$
$$
(F_{2n}y)_{n+k} = (F_ny_e)_k - (D_nF_ny_o)_k = \mathcal{F}_n(\{y_{2j}\})_k - w_{2n}^{-k}\mathcal{F}_n(\{y_{2j+1}\})_k.
$$

\end{enumerate}
\end{document}