
\documentclass{article}%
\usepackage{amsmath}
\usepackage{graphicx}
\usepackage{amsfonts}%
\usepackage{amssymb}


\setlength{\topmargin}{-0.75in}
\setlength{\textheight}{9.25in}
\setlength{\oddsidemargin}{0.0in}
\setlength{\evensidemargin}{0.0in}
\setlength{\textwidth}{6.5in}
\def\labelenumi{\arabic{enumi}.}
\def\theenumi{\arabic{enumi}}
\def\labelenumii{(\alph{enumii})}
\def\theenumii{\alph{enumii}}
\def\p@enumii{\theenumi.}
\def\labelenumiii{\arabic{enumiii}.}
\def\theenumiii{\arabic{enumiii}}
\def\p@enumiii{(\theenumi)(\theenumii)}
\def\labelenumiv{\arabic{enumiv}.}
\def\theenumiv{\arabic{enumiv}}
\def\p@enumiv{\p@enumiii.\theenumiii}
\pagestyle{plain}
\setcounter{secnumdepth}{0}
\newtheorem{theorem}{Theorem}
\newtheorem{acknowledgement}[theorem]{Acknowledgement}
\newtheorem{algorithm}[theorem]{Algorithm}
\newtheorem{axiom}[theorem]{Axiom}
\newtheorem{case}[theorem]{Case}
\newtheorem{claim}[theorem]{Claim}
\newtheorem{conclusion}[theorem]{Conclusion}
\newtheorem{condition}[theorem]{Condition}
\newtheorem{conjecture}[theorem]{Conjecture}
\newtheorem{corollary}[theorem]{Corollary}
\newtheorem{criterion}[theorem]{Criterion}
\newtheorem{definition}[theorem]{Definition}
\newtheorem{example}[theorem]{Example}
\newtheorem{exercise}[theorem]{Exercise}
\newtheorem{lemma}[theorem]{Lemma}
\newtheorem{notation}[theorem]{Notation}
\newtheorem{problem}[theorem]{Problem}
\newtheorem{proposition}[theorem]{Proposition}
\newtheorem{remark}[theorem]{Remark}
\newtheorem{solution}[theorem]{Solution}
\newtheorem{summary}[theorem]{Summary}
\newenvironment{proof}[1][Proof]{\textbf{#1.} }{\ \rule{0.5em}{0.5em}}

\newcommand{\lv}{\lVert}
\newcommand{\rv}{\rVert}

\begin{document}

\title{Theoretical Numerical Analysis, Assignment 7}
\author{Chuan Lu}
\date{\today}
\maketitle

\begin{enumerate}

\item Problem 6.1.1

By Taylor expansion,
$$
af(x+h)+bf(x)+cf(x-h) = (a+b+c)f(x) + h(a-c)f'(x) + \frac{1}{2}h^2(a+c)f''(x) + \frac{1}{6}h^3(a-c)f'''(x) + O(h^4).
$$
In order to make it be an approximation of $f'(x)$, we must have
$$
a+b+c = 0, \ h(a-c) = 1.
$$
There are three variables with two equations, so we can further let
$$
a+c = 0.
$$
Then, by letting
$$
a = \frac{1}{2h}, b = 0, c = -\frac{1}{2h},
$$
we notice $a-c\ne 0$, so
$$
|af(x+h)+bf(x)+cf(x-h)-f'(x)| \le O(h^3).
$$

\item Problem 6.1.2

The conditions now become
$$
a+b+c = 0, \ a-c = 0, \ \frac{1}{2}h^2(a+c) = 1.
$$
There is one exact solution 
$$
a = c = \frac{1}{h^2}, \ b = -\frac{2}{h^2}.
$$
Notice
$$
a-c = 0, \ \frac{1}{24}h^2(a+c)\ne 0,
$$
so
$$
|af(x+h)+bf(x)+cf(x-h)-f''(x)| \le O(h^4).
$$

\item Problem 6.2.2

\newcommand{\dt}{\Delta t}
\newcommand{\dx}{\Delta x}

The operator is defined as
$$
(1+2r)C(\dt)v(x) = r(C(\dt)v(x+\dx)+C(\dt)v(x-\dx)) + (1-2r)v(x) + r(v(x+\dx) + v(x-\dx)),
$$
where $r = \frac{\nu \dt}{2\dx^2}$.

Then we have
$$
\lv C(\dt)v\rv \le \frac{2r}{1+2r}\lv C(\dt)v\rv + \frac{1-2r}{1+2r}\lv v\rv + \frac{2r}{1+2r}\lv v\rv,
$$
which means
$$
\frac{1}{1+2r}\lv C(\dt) v\rv \le \frac{1}{1+2r}\lv v\rv.
$$
Hence,
$$
\{C(\dt)\}\le 1
$$
is uniformly bounded. We also notice
$$
\lv C(\dt)^m\rv \le 1
$$
for any $m$, so by Lax equivalence thm, Crank-Nicolson scheme is consistent.

\item Problem 6.3.1

When $bc = 0$, the eignevalues are just $\lambda_j = a  $. Now we let $Q = aI + \sqrt{bc}D^{-1}\Lambda D $ as suggested by the hint, then $\lambda_j = a+\sqrt{bc}\hat{\lambda}_j $, where $\hat{\lambda}$ are the eigenvalues of $\Lambda$. 

We can see that $\Lambda = \Lambda_n $ satisfies
$$
\det(\lambda I - \Lambda_n) = \lambda \det(\lambda I - \Lambda_{n-1}) - \det(\lambda I - \Lambda_{n-2}),
$$
so by the roots of Chebyshev polynomials of the second kind, we know $\hat{\lambda}_j = 2\cos(\frac{j\pi}{N+1}) $. 

\item Problem 6.3.2



\end{enumerate}
\end{document}