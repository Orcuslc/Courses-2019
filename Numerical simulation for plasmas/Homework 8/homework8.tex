\documentclass{article}
\usepackage{fancyhdr}
\usepackage{extramarks}
\usepackage{amsmath}
\usepackage{amsthm}
\usepackage{amsfonts}
\usepackage{tikz}
\usepackage[plain]{algorithm}
\usepackage{algpseudocode}

\begin{document}
\author{Chuan Lu}
\title{PHYS:5905 Homework 8}
\maketitle

\medskip

\begin{enumerate}

\item HW 8a, Problem 1

\begin{enumerate}

\item[(c)]

Let $u_{j}^n = \xi^ne^{ikj\Delta x} $, then 
$$
\frac{1}{2\Delta t} (\xi^{n+1}e^{ikj\Delta x} - \xi^{n-1}e^{ikj\Delta x}) = -c\frac{1}{2\Delta x} (\xi^{n}e^{ik(j+1)\Delta x} - \xi^{n}e^{ik(j-1)\Delta x}).
$$
Then
$$
\xi^2 + c\frac{\Delta t}{\Delta x} 2i\sin k\Delta x\xi -1 = 0, 
$$
we have 
$$
\xi = -c\frac{\Delta t}{\Delta x} i\sin k\Delta x \pm \sqrt{1-c^2\frac{\Delta t^2}{\Delta x^2}\sin^2 k\Delta x}.
$$

\item[(d)]
When 
$$
1-c^2\frac{\Delta t^2}{\Delta x^2}\sin^2 k\Delta x \ge 0,
$$
$$
|\xi| = \sqrt{c^2\frac{\Delta t^2}{\Delta x^2}\sin^2 k\Delta x + 1-c^2\frac{\Delta t^2}{\Delta x^2}\sin^2 k\Delta x} = 1,
$$
so the leapfrog scheme is stable when 
$$
c\frac{\Delta t}{\Delta x} \le 1.
$$
On the other hand, when $c\frac{\Delta t}{\Delta x} > 1$, then for some $k > 0$, 
$$
1-c^2\frac{\Delta t^2}{\Delta x^2}\sin^2 k\Delta x < 0,
$$
then
$$
|\xi| = |c\frac{\Delta t}{\Delta x}\sin k\Delta x| + |\sqrt{c^2\frac{\Delta t^2}{\Delta x^2}\sin^2 k\Delta x - 1}| > 1,
$$
so the algorithm is unstable. Hence, the stability condition is
$$
c\frac{\Delta t}{\Delta x} \le 1.
$$

\end{enumerate}

\item HW 8a, Problem 2

\begin{enumerate}
\item[(e)]
The result is shown in Figure \ref{Problem 2(e)}.

\begin{figure}[ht]
\centering
\includegraphics[scale=0.8]{2e.png}
\caption{$u(x) $ at $t = 0, 0.25, 0.5, 0.75, 1.0$ with $n_x = 128 $, $\Delta t = \frac{1}{128}$.}
\label{Problem 2(e)}
\end{figure}

\item[(f)]
The result is shown in Figure \ref{Problem 2(f)}.

\begin{figure}[h]
\centering
\includegraphics[scale=0.8]{2f.png}
\caption{$u(x)$ at $t = 0$, and at $t = 1$ with $\Delta t = \frac{1}{128}, \frac{1}{256}, \frac{1}{512}, \frac{1}{1024}$, while $n_x = 128 $.}
\label{Problem 2(f)}
\end{figure}

\item[(g)]
Since the stability condition for leapfrog method is
$$
|c|\frac{\Delta t}{\Delta x} \le 1,
$$
which is satisfied for the different configurations, so the algorithm is stable.

\item[(h)]
Now we pick $n_x = 128$ and $\Delta t = 1/100$, and the plot is shown in Figure \ref{Problem 2(h)}.

\begin{figure}[h]
\centering
\includegraphics[scale=0.8]{2h.png}
\caption{$u(x)$ at $t = 0, 0.25, 0.5, 0.75, 0.8125$ with $\Delta t = \frac{1}{100}$ and $n_x = 128 $.}
\label{Problem 2(h)}
\end{figure}

\end{enumerate}

\item HW 8b, Problem 1

\begin{enumerate}
\item[(d)]
The plot is shown in Figure \ref{Problem 3(d)}.

\begin{figure}[h]
\centering
\includegraphics[scale=0.8]{3d.png}
\caption{$u(x)$ at $t = 100$ with $\Delta t = \frac{1}{128}, \frac{1}{192}, \frac{1}{256}, \frac{1}{512}, \frac{1}{1024}, \frac{1}{2048}$ and $n_x = 128 $.}
\label{Problem 3(d)}
\end{figure}

\item[(e)]
The graph is shown in Figure \ref{Problem 3(e)}, where the fit is done by \texttt{scipy.optimize.curve\_fit}.

\begin{figure}[h]
\centering
\includegraphics[scale=0.8]{3e.png}
\caption{$\delta/\pi$ vs. $c\Delta t/\Delta x$ for $\Delta t = \frac{1}{128}, \frac{1}{192}, \frac{1}{256}, \frac{1}{512}, \frac{1}{1024}, \frac{1}{2048}$ and $n_x = 128 $.}
\label{Problem 3(e)}
\end{figure}

\item[(f)]
The plot is in Figure \ref{Problem 3(f)}.

\begin{figure}[h]
\centering
\includegraphics[scale=0.8]{3f.png}
\caption{$\delta/\pi$ vs. $c\Delta t/\Delta x$ for $\Delta t = \frac{1}{16}, \frac{1}{24}, \frac{1}{32}, \frac{1}{64}, \frac{1}{128}, \frac{1}{256}$ and $n_x = 16 $.}
\label{Problem 3(f)}
\end{figure}

\item[(g)]
The plot is in Figure \ref{Problem 3(g)}.

\begin{figure}[h]
\centering
\includegraphics[scale=0.8]{3g.png}
\caption{$u_f/u_0 $ vs. $c\Delta t/\Delta x$ for $\Delta t = \frac{1}{16}, \frac{1}{24}, \frac{1}{32}, \frac{1}{64}, \frac{1}{128}, \frac{1}{256}$ and $n_x = 16 $.}
\label{Problem 3(g)}
\end{figure}

\end{enumerate}

\end{enumerate}

\end{document}