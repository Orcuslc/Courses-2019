\documentclass{article}
\usepackage{fancyhdr}
\usepackage{extramarks}
\usepackage{amsmath}
\usepackage{amsthm}
\usepackage{amsfonts}
\usepackage{tikz}
\usepackage[plain]{algorithm}
\usepackage{algpseudocode}

\begin{document}
\author{Chuan Lu}
\title{PHYS:5905 Homework 10}
\maketitle

\medskip

\begin{enumerate}

\item Problem 1

\begin{enumerate}

\item The output is as follows.

Hello World! I am processor 4 of 16 processors.

Hello World! I am processor 8 of 16 processors.

Hello World! I am processor 9 of 16 processors.

Hello World! I am processor 10 of 16 processors.

Hello World! I am processor 11 of 16 processors.

Hello World! I am processor 12 of 16 processors.

Hello World! I am processor 13 of 16 processors.

Hello World! I am processor 15 of 16 processors.

Hello World! I am processor 0 of 16 processors.

Hello World! I am processor 1 of 16 processors.

Hello World! I am processor 3 of 16 processors.

Hello World! I am processor 5 of 16 processors.

Hello World! I am processor 6 of 16 processors.

Hello World! I am processor 7 of 16 processors.

Hello World! I am processor 14 of 16 processors.

Hello World! I am processor 2 of 16 processors.

\item My method is to use \texttt{MPI\_Barrier()} to block $(i+1)^{th}, \cdots, n^{th} $ processors until the $i^{th} $ processor has executed its code.

\end{enumerate}

\item Problem 2

The plot is shown as follows. I excluded all initialization time in time calculation.

\begin{figure}
\centering
\includegraphics[scale=0.8]{2c.png}
\caption{Computation time vs. number of processors.}
\label{1}
\end{figure}


\end{enumerate}

\end{document}