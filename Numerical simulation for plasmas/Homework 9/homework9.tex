\documentclass{article}
\usepackage{fancyhdr}
\usepackage{extramarks}
\usepackage{amsmath}
\usepackage{amsthm}
\usepackage{amsfonts}
\usepackage{tikz}
\usepackage[plain]{algorithm}
\usepackage{algpseudocode}

\begin{document}
\author{Chuan Lu}
\title{PHYS:5905 Homework 9}
\maketitle

\medskip

\begin{enumerate}

\item HW 9a, Problem 1

\begin{enumerate}
\item Problem (f)

The plot of of $u(x)$ is shown in Figure \ref{Problem 1(f)}.

\begin{figure}[ht]
\centering
\includegraphics[scale=0.8]{1f.png}
\caption{$u'(x) $ at $t = 0, 0.25, 0.5, 0.75, 1.0$ with $n_x = 128 $, $\Delta t = \frac{1}{512}$.}
\label{Problem 1(f)}
\end{figure}

\item Problem (h)

The plot of $J_{\pm} $ at $t' = 0.25$ and $t' = 0.5$ is shown in Figure \ref{Problem 1(h)}. $J_{-} $ is approximately a constant.

\begin{figure}[ht]
\centering
\includegraphics[scale=0.8]{1h.png}
\caption{$J_{\pm}(x) $ at $t = 0.25, 0.5$.}
\label{Problem 1(h)}
\end{figure}

\item Problem (i)

Since $J_{+} $ is not constant, we advance $x'$ at speed $u'+c_s' $ over $t'$. The result is shown in Figure \ref{Problem 1(i)}.

\begin{figure}[ht]
\centering
\includegraphics[scale=0.8]{1i.png}
\caption{numerical and analytical solution of $u $ at $t = 0.25, 1.0$.}
\label{Problem 1(i)}
\end{figure}

\item Problem (j)

The result is shown in Figure \ref{Problem 1(j)}. There might be something wrong in my code, since there should already be a shock wave at $t = 2.0$.

\begin{figure}[ht]
\centering
\includegraphics[scale=0.8]{1j.png}
\caption{numerical and analytical solution of $u $ at $t = 2.0$.}
\label{Problem 1(j)}
\end{figure}

\end{enumerate}



\end{enumerate}

\end{document}