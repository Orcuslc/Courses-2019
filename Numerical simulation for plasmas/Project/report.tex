\documentclass{article}
\usepackage[margin=2cm]{geometry}
\usepackage{fancyhdr}
\usepackage{extramarks}
\usepackage{amsmath}
\usepackage{amsthm}
\usepackage{amsfonts}
\usepackage{tikz}
\usepackage[plain]{algorithm}
\usepackage{algpseudocode}

\begin{document}
\author{Chuan Lu}
\title{PHYS:5905 Final Project}
\maketitle

\medskip

\begin{enumerate}
\item MPI/OpenMP hybrid code: add OpenMP support for the the MPI-enabled HYDRO code.

\item Setup:
We run the simulation on a $1024\times 1024$ grid on $[0, 1]\times [0, 1]$, with number of timesteps being 4000 and final time $T = 1.0$. Since the next time step requires information of the current and the previous steps, we only add openmp in each timesteps, but not between timesteps.

\item Validation:
Compared the new outputs with the original ones using $\texttt{diff}$ (with grid size $128\times 128$). All outputs are exactly the same. We also compared the output of different settings of parallelization with grid size $1024\times 1024$, and the outputs are also the same.

\item Time: We test the program on two 32-core nodes on Argon, with OMP\_NUM\_THREADS ranging from 32 to 1 while keeping the product of threads and MPI tasks at 64. Figure \ref{1} shows the computational time of the options. Using OpenMP on each MPI task increases the total time cost when the grid size is not very large. 

\begin{figure}[h]
\centering
\includegraphics[scale=0.6]{test1.jpg}
\caption{Running time vs. OMP threads}
\label{1}
\end{figure}

\end{enumerate}


\end{document}